\documentclass{article}

\usepackage{xcolor}
\usepackage[T1]{fontenc}
\usepackage{pgffor}
\usepackage{todonotes}
\usepackage{minted}
\usepackage{graphicx}
\usepackage{minted}
\usemintedstyle{vs}
\usepackage{fancyhdr}
\usepackage{hyperref}
\usepackage{tcolorbox}
\usepackage{etoolbox}
\usepackage[margin=1.2in]{geometry}
\usepackage[ngerman]{babel} 
\usepackage{inconsolata}
\usepackage{bookmark}
\usepackage{wrapfig} 
\usepackage{lipsum}
\usepackage[autolang=other, backend=biber]{biblatex}
\addbibresource{refs.bib}

\hypersetup{
  colorlinks=true,
  linkcolor=blue,
  filecolor=magenta,
  citecolor=blue,
  urlcolor=blue,
}

\renewcommand*\familydefault{\ttdefault} %% Only if the base font of the document is to be typewriter style

\newcommand{\topic}[1]{\tcbox[on line,arc=4pt,colframe=white,boxrule=0pt,boxsep=0pt,left=4pt,right=4pt,top=3pt,bottom=2pt,colback=gray!30]{#1}}

\newcommand{\br}{\linebreak\linebreak}

\newcommand{\link}[2][]{%
    \ifblank{#1}{%
        \hyperref[sec:#2]{#2}%
    }{%
        \hyperref[sec:#1]{#2}%
    }%
}

\newenvironment{code}
  {%
    \begin{tcolorbox}[
        colback=white,
        colframe=black,
        boxrule=0.5pt,
        arc=4pt,
        left=2mm,
        right=2mm,
        top=2mm,
        bottom=2mm,
      ]
      }{% 
    \end{tcolorbox}
    }


\newcommand{\cmt}[1]{\textcolor{gray!60}{\textit{/* #1 */}}}

\newcommand{\qt}[1]{„#1“}

% Define a custom command for colored boxes around words
\newcommand{\topics}[1]{%
  \linebreak
  \linebreak
  \foreach \word in {#1} {%
    \topic{\word}%
  }%
  \linebreak
}


\title{Exposé für eine Bachelorarbeit}
\author{Max Richter}


\begin{document}

\pagestyle{fancy}
\fancyhead{} % clear all header fields
\fancyhead[RO,LE]{\textbf{WebAssembly-basierten Visualen Programmiersprache}}
\fancyfoot{} % clear all footer fields
\fancyfoot[LE,RO]{\thepage}
\fancyfoot[LO,CE]{\href{https://github.com/jim-fx/bachelor}{github.com/jim-fx/bachelor}}
\fancyfoot[CO,RE]{Max Richter}
% \href{https://max-richter.dev}

\raggedright

\maketitle
\pagebreak

{\LARGE Entwicklung einer Performanten WebAssembly-basierten Visualen Programmiersprache}
  
\section{Problemstellung}
In dieser Bachelorarbeit werde ich versuchen eine node-basierte visuelle Programmiersprache zu entwickeln, mit der Besonderheit das die einzelnen Nodes WebAssembly Module sind.
\linebreak
\linebreak
Durch die Sandboxing Capabilities von WebAssembly können diese Module einigermaßen sicher ausgeführt werden und so könnte man einen öffentlichen Marktplatz für diese Nodes erstellen.
\linebreak
\linebreak
Dies wiederum würde es erlauben ein generelles Framework bereitzustellen bei denen Node-basierte UI's für unterschiedliche Usecases und dank der cross-plattform eigenschaften von WebAssembly auch für unterschiedliche Plattformen erstellt werden können.

\section{Abgeleitete Forschungsfrage}
Diese Arbeit untersucht ob eine node-basierte visuelle Programmiersprache, bei der die Nodes WebAssembly module sind, implementiert werden kann und ob diese performant genug ist um in einem UI verwendet zu werden.

\section{Vorgehen}
\subsection{Literatur Recherche}
Als erstes werde ich eine allgemeine Literaturrecherche zu den Themen \topic{WebAssembly}, \topic{visuelle Programmiersprachen} und im speziellen \topic{node-basierte visuelle Programmiersprachen} durchführen.

\subsection{Recherche von bestehenden Lösungen}

Node-basierte visuelle Programmiersprachen sind nichts neues und es gibt viele bestehende Lösungen.

\subsection{Implementierung}

Der Grpßteil der Arbeit wird in der Implementierung liegen. 

\subsection{Evaluation}

Im letzten Schritt werde ich die Implementierung auf ihrer performance analysieren.

% <Skizzieren Sie in 2-3 Absätzen, wie Sie in vorgehen möchten, um die Forschungsfrage der Arbeit zu beantworten. Hier geht es also um eine Art informellen „Projektplan“, bei dem Sie die einzelnen Arbeitsschritte beschreiben. Reihenfolge und Umfang der einzelnen Schritte können, müssen aber nicht festgelegt sein.>

\section{Formales}
Geplantes Startdatum: 08.01.2024
\linebreak
Sperrvermerk geplant: nein
\linebreak
Begründung: n.z.

\section{Externer Kooperationspartner}
n.z.

\end{document}
