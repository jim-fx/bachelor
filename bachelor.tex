\documentclass[ngerman]{article}

\usepackage{xcolor}
\usepackage{inconsolata}
\usepackage[T1]{fontenc}
\usepackage{pgffor}
\usepackage{graphicx}
\usepackage{fancyhdr}
\usepackage{hyperref}
\usepackage{tcolorbox}
\usepackage[margin=1.2in]{geometry}
\usepackage{biblatex}
\addbibresource{refs.bib}

\hypersetup{
  colorlinks=true,
  linkcolor=blue,
  filecolor=magenta,
  urlcolor=blue,
}

\renewcommand*\familydefault{\ttdefault} %% Only if the base font of the document is to be typewriter style

\newcommand{\topic}[1]{\tcbox[on line,arc=4pt,colframe=white,boxrule=0pt,boxsep=0pt,left=4pt,right=4pt,top=3pt,bottom=2pt,colback=gray!30]{#1}}

% Define a custom command for colored boxes around words
\newcommand{\topics}[1]{%
  \linebreak
  \linebreak
  \foreach \word in {#1} {%
    \topic{\word}%
  }%
  \linebreak
}


\title{Bachelorarbeit}
\author{Max Richter}

\begin{document}

\pagestyle{fancy}
\fancyhead{} % clear all header fields
\fancyhead[RO,LE]{\textbf{WebAssembly-basierte visuelle Programmiersprache}}
\fancyfoot{} % clear all footer fields
\fancyfoot[LE,RO]{\thepage}
\fancyfoot[LO,CE]{\href{https://github.com/jim-fx/bachelor}{github.com/jim-fx/bachelor}}
\fancyfoot[CO,RE]{Max Richter}

\raggedright

\maketitle
\pagebreak

\tableofcontents

\pagebreak

\section{Einleitung}
\subsection{Hintergrund}
\subsection{Problemstellung}
\subsection{Zielsetzung}
\subsection{Forschungsfragen}

\section{Theoretischer Rahmen}
\subsection{Literatur Recherche}
\subsection{WebAssembly}
WebAssembly (WASM) ist ein Bytecode-Format für eine stack-basierte virtuelle Maschine. Es wurde von der WebAssembly Working Group des World Wide Web Consortium (W3C) entwickelt und ist seit 2017 ein offizieller Web-Standard \cite{Haas2017}.
\linebreak
\linebreak
Es wird hauptsächlich als Kompilierungsziel für Programmiersprachen wie C, C++, Rust und TypeScript verwendet, um diese in Webanwendungen zu integrieren. Im Gegensatz zu Javascript ist WebAssembly nicht Garbage-Collectet und ist außerdem stark typisiert. Dies ermöglicht es, performante und sicherere Anwendungen zu entwickeln.
\subsubsection{WASI}
\cmt{WebAssembly System Interface}
\subsubsection{WAT}
\cmt{WebAssembly Text Format}
\subsubsection{WebAssembly Component Model}
\cmt{Kind of like ES-Modules or C-Style Linking}
\subsection{Visuelle Programmiersprachen}
\subsubsection{Node-basierte visuelle Programmiersprachen}

\section{Implementierung}
\subsection{UI}
\subsection{Backend}

\section{Evaluation}
\section{Fazit}
\section{Literaturverzeichnis}


\printbibliography

\end{document}
